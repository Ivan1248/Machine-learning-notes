\documentclass{article}
\usepackage[utf8]{inputenc}
% font
\usepackage{lmodern}
\renewcommand{\familydefault}{\sfdefault}  % sans-serif main

%\usepackage[cm]{sfmath}  % bolje nego mathastext
%\SetSymbolFont{largesymbols}{normal}{OMX}{iwona}{m}{n}

%\usepackage[italic]{mathastext}  % sfmath je bolje (manji indeksi)

%\usepackage{inconsolata}					% sans-serif monospace
\usepackage[scaled]{beramono}				% sans-serif monospace


%\usepackage[math]{iwona}
%\usepackage[math]{kurier}

%\newcommand*{\scale}[2][4]{\scalebox{#1}{$#2$}} % \Scale[0.5]{y = \sin^2 x}
%\usepackage{scalerel}

\usepackage[T1]{fontenc}  % accented characters, copy from pdf, ...

% theorem, definition
\usepackage[english]{babel}
\newtheorem{theorem}{Theorem}
\newtheorem{definition}{Definition}


\raggedright	% no right alignment
\raggedbottom   % no vertical stretching

% Captions
\usepackage{caption}
\captionsetup{%
	justification=raggedright,
}

\usepackage{etoolbox}
\makeatletter
	\patchcmd{\@dottedtocline}
	{\rightskip\@tocrmarg}
	{\rightskip\@tocrmarg plus 4em \hyphenpenalty\@M}
	{}{}
\makeatother
\setlength{\parindent}{1em}	 % uvlačenje ulomaka
\usepackage{indentfirst}	 % uvlačenje prvog ulomka
\setlength{\parskip}{0.5em}	 % razmak između ulomaka

\usepackage{listings}  % listings
\renewcommand{\lstlistingname}{Ispis}

\usepackage[multiple, bottom]{footmisc}	 % višestruke fusnote, poslije slika/tablica

\usepackage[hidelinks]{hyperref}
\renewcommand*{\UrlFont}{\footnotesize}

% colors

\usepackage{xcolor}
\usepackage{color}

\hypersetup{
	colorlinks,
	linkcolor={blue!60!green!50!black},  % xcolor package
	citecolor={green!40!black},
	urlcolor={blue!75!green!30!black}
}
\definecolor{bluekeywords}{rgb}{0.13,0.13,1}  % color package
\definecolor{greencomments}{rgb}{0,0.5,0}
\definecolor{redstrings}{rgb}{0.9,0,0}

%\newtheorem{theorem}{Teorem}[section]
%\newtheorem{definition}{Definicija}[section]

\usepackage{enumitem} \setlist[enumerate,1]{topsep=0pt,itemsep=0pt,partopsep=0pt}
\setlist[itemize,1]{topsep=0pt,itemsep=0pt,partopsep=0pt}

 
\DeclareTextFontCommand{\emphasize}{\bfseries}

\newcommand{\say}[1]{"\textit{#1}}

% redefinition of left and right to make spacing consistent
\let\originalleft\left
\let\originalright\right
\def\left#1{\mathopen{}\originalleft#1}
\def\right#1{\originalright#1\mathclose{}}

\usepackage{amsmath}
\usepackage{amssymb}  % loads amsfonts

\usepackage{mathtools}  % \coloneqq
%\usepackage{bm}
%\usepackage[utopia]{mathdesign}
\usepackage[OMLmathsfit]{isomath}  % \DeclareMathAlphabet ...

\usepackage{mathtools}  % smashoperator
\usepackage{upgreek}

%\usepackage{commath}  % calculus, perentheses
% https://tex.stackexchange.com/questions/135944/commath-and-ifinner/135985#135985
%
\DeclareMathOperator{\dif}{d \!}
\DeclareMathOperator{\Dif}{D \!}

\makeatletter
\newcommand{\spx}[1]{%
	\if\relax\detokenize{#1}\relax
	\expandafter\@gobble
	\else
	\expandafter\@firstofone
	\fi
	{^{#1}}%
}
\makeatother

\newcommand\pd[3][]{\frac{\partial\spx{#1}#2}{\partial#3\spx{#1}}}
\newcommand\tpd[3][]{\tfrac{\partial\spx{#1}#2}{\partial#3\spx{#1}}}
\newcommand\dpd[3][]{\dfrac{\partial\spx{#1}#2}{\partial#3\spx{#1}}}

\newcommand{\md}[6]{\frac{\partial\spx{#2}#1}{\partial#3\spx{#4}\partial#5\spx{#6}}}
\newcommand{\tmd}[6]{\tfrac{\partial\spx{#2}#1}{\partial#3\spx{#4}\partial#5\spx{#6}}}
\newcommand{\dmd}[6]{\dfrac{\partial\spx{#2}#1}{\partial#3\spx{#4}\partial#5\spx{#6}}}

\newcommand{\od}[3][]{\frac{\dif\spx{#1}#2}{\dif#3\spx{#1}}}
\newcommand{\tod}[3][]{\tfrac{\dif\spx{#1}#2}{\dif#3\spx{#1}}}
\newcommand{\dod}[3][]{\dfrac{\dif\spx{#1}#2}{\dif#3\spx{#1}}}

\newcommand{\genericdel}[4]{%
	\ifcase#3\relax
	\ifx#1.\else#1\fi#4\ifx#2.\else#2\fi\or
	\bigl#1#4\bigr#2\or
	\Bigl#1#4\Bigr#2\or
	\biggl#1#4\biggr#2\or
	\Biggl#1#4\Biggr#2\else
	\left#1#4\right#2\fi
}
\newcommand{\del}[2][-1]{\genericdel(){#1}{#2}}
\newcommand{\set}[2][-1]{\genericdel\{\}{#1}{#2}}
\let\cbr\set
\let\event\set
\newcommand{\sbr}[2][-1]{\genericdel[]{#1}{#2}}
\let\intoo\del
\let\intcc\sbr
\newcommand{\intoc}[2][-1]{\genericdel(]{#1}{#2}}
\newcommand{\intco}[2][-1]{\genericdel[){#1}{#2}}
\newcommand{\eval}[2][-1]{\genericdel.|{#1}{#2}}
\newcommand{\envert}[2][-1]{\genericdel||{#1}{#2}}
\let\abs\envert
\newcommand{\sVert}[1][0]{%
	\ifcase#1\relax
	\rvert\or\bigr|\or\Bigr|\or\biggr|\or\Biggr
	\fi
}
\newcommand{\enVert}[2][-1]{\genericdel\|\|{#1}{#2}}
\let\norm\enVert
\newcommand{\fullfunction}[5]{%
	\begin{array}{@{}r@{}l@{}}
		#1 \colon #2 &{}\longrightarrow{}  #3 \\
		#4 &{}\longmapsto{} #5
	\end{array}
}
%
%


\usepackage{stmaryrd}  % \llbracket for \rrbracket, ...


%\usepackage[croatian]{babel}		% teorem
%\newtheorem{definition}{Definicija}[section]
%\newtheorem{theorem}{Teorem}[section]
%\newtheorem{corollary}{Korolar}[theorem]

\DeclareMathAlphabet{\mathbbmsl}{U}{bbm}{m}{sl}
\DeclareMathAlphabet{\mathbbmb}{U}{bbm}{b}{it}
\DeclareMathAlphabet{\mathbbmssit}{U}{bbmss}{m}{it}


% common set?, distribution
\newcommand{\commset}[1]{\mathbbmb{#1}}
\newcommand{\distrib}[1]{\mathcal{#1}}

% sans-serif blackboard-bold
\newcommand{\mathsfbbit}[1]{\mathbbmssit{#1}}

% variable
\let\vec\relax
\let\set\relax
\newcommand{\vec}[1]{\mathbfit{#1}}
\newcommand{\set}[1]{\mathbbmsl{#1}}

% constant
\newcommand{\const}[1]{\mathrm{#1}}
\newcommand{\cvec}[1]{\mathbf{#1}}
\newcommand{\cset}[1]{\mathbb{#1}}

% random variable
%\newcommand\mathmakebox[3][1em][c]{\mathpalette\mathmakeboxinternal[#1][#2]{#3}}
%\newcommand\mathmakeboxinternal[3][1em][c]{\makebox[#1][#2]{$\mathsurround=0pt{#3}$}}
\usepackage{etex}
\newcommand{\nunder}[2][5]{\mathrlap{\mkern\the\numexpr#1/2mu\relax\underline{\phantom{\mathrm{#2}\mkern-#1mu}}}\mathrm{#2}}
\newcommand{\subline}[2][n]{%
    \makebox[\widthof{$#2$}/2]{}\mathclap{#2}%
    \text{\smash{\raisebox{0.04em}{% 0.04
        $\mathclap{\underline{\hphantom{#1}\vphantom{#2}}}$%
    }}}%
    \makebox[\widthof{$#2$}/2]{}%
}
\newcommand{\rvarstyle}[1]{{{\subline[i]{#1}}}}
\newcommand{\rvar}[1]{\rvarstyle{#1}}
\newcommand{\rvec}[1]{\rvarstyle{\vec{#1}}}
\newcommand{\rset}[1]{\rvarstyle{\set{#1}}}
\newcommand{\var}[1]{{\color{red}#1}}

% linear algebra
\newcommand{\transpose}{\mathsf T}
\newcommand{\tp}{\transpose}

% calculus - commath: od, pd, md, dif
% parentheses - commath: del, cbr, sbr, envert, enVert

% named functions
\DeclareMathOperator{\softplus}{softplus}
\DeclareMathOperator{\softmax}{softmax}
\DeclareMathOperator{\logistic}{\sigma}
\DeclareMathOperator{\sgn}{sgn}
\DeclareMathOperator{\diag}{diag}
\DeclareMathOperator{\ReLU}{ReLU}
\DeclareMathOperator{\modfunc}{mod}

% operators
\DeclareMathOperator*{\argmin}{arg\,min} % thin space
\DeclareMathOperator*{\argmax}{arg\,max}
\let\P\relax
\newcommand{\P}{\mathrm{P}}
\newcommand{\p}{\mathrm{p}}
\DeclareMathOperator*{\E}{\mathrm{I\kern-.282em E}}
\DeclareMathOperator*{\D}{\mathrm{I\kern-.282em D}}
\DeclareMathOperator*{\Cov}{\mathrm{Cov}}
%\let\H\relax
\renewcommand{\H}{\mathrm{H}}
\newcommand{\h}{\mathrm{h}}
\newcommand{\I}{\mathrm{I}}
\newcommand{\Dist}{\mathrm{D}}  % D is for dispersion
%\renewcommand{\dif}{\mathrm{d}}

\newcommand{\pa}{\mathrm{pa}}
\newcommand{\ch}{\mathrm{ch}}
\newcommand{\pred}{\mathrm{pred}}
\renewcommand{\succ}{\mathrm{succ}}

\newcommand{\vecfunc}{\mathrm{vec}}

% bracket operators
%\newcommand{\enangle}[1]{\mathinner{\left\langle{#1}\right\rangle}}
\newcommand{\enangle}[2][-1]{\genericdel\langle\rangle{#1}{#2}}
\newcommand{\enbbracket}[1]{{\mathinner{\left\llbracket{#1}\right\rrbracket}}}
\newcommand{\braket}[2]{\enangle{{#1}\middle|{#2}}}

% parentheses from commath redefined to improve spacing because left and right were redefined, \midmid, new \tilde and \hat that don't make parentheses bigger
\renewcommand{\del}[1]{\left(#1\right)}
\renewcommand{\sbr}[1]{\left[#1\right]}
\renewcommand{\cbr}[1]{\left\{#1\right\}}
\renewcommand{\intoo}[1]{\mathinner{\del{#1}}}
\renewcommand{\intcc}[1]{\mathinner{\sbr{#1}}}
\renewcommand{\intco}[1]{\mathinner{\left[#1\right)}}
\renewcommand{\intoc}[1]{\mathinner{\left(#1\right]}}
\newcommand{\ind}[1]{{\sbr{#1}}}
%\let\mid\relax
%\newcommand{\mid}{\;|\;}  % spaces around not reduced in index
\newcommand{\midmid}{\;\middle|\;}
\let\oldhat\hat
\renewcommand{\hat}[1]{\vphantom{#1}\smash[t]{\oldhat{#1}}}
\let\oldtilde\tilde
\renewcommand{\tilde}[1]{\vphantom{#1}\smash[t]{\oldtilde{#1}}}
%\renewcommand{\tilde}[1]{\stackrel{\sim}{\smash{\mathcal{#1}}\rule{0pt}{1.1ex}}}
%\renewcommand{\tilde}[1]{\overset{\sim}{#1}}}
\let\oldwidetilde\widetilde
\renewcommand{\widetilde}[1]{\vphantom{#1}\smash[t]{\oldwidetilde{#1}}}

% special
%\newcommand{\funcdef}[3]{#1 \in (#2 \to #3)}
\newcommand{\funcdef}[3]{#1 \colon #2 \to #3}
\newcommand{\Dkl}[2]{\Dist_\mathrm{KL}\del{#1\;\middle\|\;#2}}
\newcommand{\C}{\cset{C}}
\newcommand{\R}{\cset{R}}
\newcommand{\Z}{\cset{Z}}
\newcommand{\N}{\cset{N}}
\newcommand{\dirac}{\updelta}

\newcommand\concat{\mathbin{+\mkern-10mu+}}

\newcommand{\TP}{\mathit{TP}}
\newcommand{\FP}{\mathit{FP}}
\newcommand{\TN}{\mathit{TN}}
\newcommand{\FN}{\mathit{FN}}
\newcommand{\FPR}{\mathit{FPR}}
\newcommand{\TPR}{\mathit{TPR}}
\newcommand{\AUROC}{\mathit{AUROC}}
\newcommand{\AUPR}{\mathit{AUPR}}
\newcommand{\AP}{\mathit{AP}}
\newcommand{\IoU}{\mathit{IoU}}
\newcommand{\mIoU}{\mathit{mIoU}}

% operators and relations
\newcommand{\bidot}{\mkern1.5mu{..}\mkern1.5mu}
%\newcommand\sheq{\mkern1.5mu{=}\mkern1.5mu}
%\renewcommand{\dots}{...}


% no vertical space stretching around display math
\usepackage[nodisplayskipstretch]{setspace}

\newenvironment{talign}
{\let\displaystyle\textstyle\align}
{\endalign}
\newenvironment{talign*}
{\let\displaystyle\textstyle\csname align*\endcsname}
{\endalign}


\usepackage{dashbox}%
%\newcommand\dashedbox[1][H]{\setlength{\fboxsep}{0pt}\setlength{\dashlength}{4pt}\setlength{\dashdash}{2pt} \dbox{\phantom{#1}}}
\newcommand\phbox[1][\phantom{:}]{\setlength{\fboxsep}{0.5pt}\setlength{\dashlength}{4pt}\setlength{\dashdash}{3pt} \raisebox{2pt}{\dbox{$\phantom{:}_{#1}$}}}
%\newcommand\dashedbox[1][H]{\setlength{\fboxsep}{0pt}\dbox{\phantom{#1}}}

\usepackage{booktabs}  % from the tamplate; table quality enhancement

\usepackage{multirow}
\usepackage{tabularx}
\usepackage{tabu}
\usepackage{makecell}

\newcolumntype{P}[1]{>{\raggedright\let\newline\\\arraybackslash\hspace{0pt}}p{#1}}

\renewcommand{\arraystretch}{1.1}
%\usepackage{floatrow} 				% centriranje svih slika
\usepackage{float}					% figure [H]
\usepackage{graphicx} 				% includegraphics
\usepackage{caption}			% subfigure
\usepackage{subcaption}			% subfigure
\usepackage[export]{adjustbox} 	% http://ctan.org/pkg/adjustbox

\usepackage[section]{placeins}  % [section] for \FloatBarrier before every section 

\graphicspath{ {./figures/} }		% mapa sa slikama
%\let\oldincludegraphics\includegraphics
%\renewcommand{\includegraphics}[2][]{\oldincludegraphics[#1,max width=0.9\linewidth]{#2}}

%\usepackage{flafter} % floats after the first reference

\usepackage{tikz} 					% dijagrami
\usepackage{pgfplots}
\usetikzlibrary{fit,automata,arrows,positioning,calc,petri,topaths,arrows.meta}

%\usetikzlibrary{external}
%\tikzexternalize[prefix=figures/tikz/, shell escape=-enable-write18]
%TexStudio Configuration\commands\pdflatex:
%"pdflatex.exe -src -interaction=nonstopmode --shell-escape %.tex

\pgfplotsset{every axis/.append style={
		axis x line=middle,    	% put the x axis in the middle
		axis y line=middle,    	% put the y axis in the middle
		axis line style={->},  	% arrows on the axis
		xlabel={$x$},          	% default put x on x-axis
		ylabel={$y$},          	% default put y on y-axis
		samples=100,
		axis equal,
}} % axis style

\tikzset{
	>={Triangle[length=1.8mm,width=1.2mm]},
	dedge/.style={arrows=->, black, thick},
}
% PGMs
\tikzstyle{textnode} = [minimum size=5mm, node distance=9mm]
\tikzstyle{pnode} = [circle, minimum size=10mm, thick, draw=black, node distance=9mm]
\tikzstyle{greypnode} = [pnode, fill=black!5]
% https://github.com/jluttine/tikz-bayesnet
\tikzstyle{wrap} = [inner sep=0pt, fit=#1]
\tikzstyle{plate} = [draw, rectangle, rounded corners=0.5ex, thick, fit=#1]
\tikzstyle{caption} = [font=\footnotesize, node distance=0]
\tikzstyle{plate caption} = [caption, node distance=0, inner sep=0pt,
below left=5pt and 0pt of #1.south east]
\newcommand{\plate}[4][]{
	\node[wrap=#3] (#2-wrap) {};
	\node[plate caption=#2-wrap] (#2-caption) {#4};
	\node[plate=(#2-wrap)(#2-caption), #1] (#2) {};
}
% ANNs
\tikzstyle{nnode} = [circle, minimum size=10mm, thick, draw=black, node distance=5mm]
\tikzstyle{nrect} = [rectangle, minimum size=7mm, thick, draw=black, node distance=5mm, rounded corners=0.1ex]
\usepackage{dirtree}

\usepackage[]{algorithmic}

\usepackage{setspace}  % line spacing

\usepackage{pdfpages} % inclusion of external pdf pages

\usepackage[toc]{appendix}
\usepackage[sort&compress,round,comma,authoryear]{natbib}




\title{Machine learning notes}
\author{}
\date{}

\usepackage[symbols,nopostdot,nonumberlist,section]{glossaries-extra}

%\renewcommand{\glossarypreamble}{\footnotesize}

\newglossarystyle{supergroup}{%
	\setglossarystyle{super}%
	\renewcommand*{\glsgroupskip}{}%
	\renewcommand{\glossentry}[2]{%
		\tabularnewline%
		\multicolumn{2}{p{\textwidth}}{%
			\raggedright\glsentryitem{##1}\glstarget{##1}{\glossentryname{##1}}%
		}% 
		\vspace{2mm}%
		\tabularnewline%
	}%
	\renewcommand{\subglossentry}[3]{%
		\glssubentryitem{##2}%
		\glstarget{##2}{\glossentryname{##2}}&%
		\raggedright\glossentrydesc{##2}\glspostdescription\space##3\tabularnewline%
	}%
}
\newcommand{\test}[1]{ \def\tst{#1} \ifx\tst\empty \typeout{optional argument was omitted} \else \typeout{optional argument was given: '#1'} \fi}
\newcommand{\glsgroup}[3]{%
	\newglossaryentry{#1}{type=symbols, name={{\large \textbf{#2}} \def\temp{#3}\ifx\temp\empty\else\vspace{2mm}\newline #3\fi}, description={}}
}
\newcommand{\glsent}[4]{\newglossaryentry{{#1:#2}}{sort={#2},type=symbols,name={#3},description={#4},parent={#1}}}
\newcommand{\glsentm}[4]{\glsent{#1}{#2}{\ensuremath{\displaystyle#3}}{#4}}

%\setglossarypreamble[symbols]{Ovaj odjeljak sadrži popis velikog broja oznaka koje se koriste u ovom radu. Za neke skupine oznaka napisana su kratka objašnjenja koja dodatno pojašnjavaju i opravdavaju neke oznake. Pojmovi koje označavaju neke oznake detaljnije su objašnjeni u poglavlju~\ref{chap:osnovni-pojmovi}.}

% Objekti
\glsgroup{o}{Objects}
{Variables are generally denoted by italic serif letters. Most constants are denoted by upright serif letters. Random variables are underlined. Vectors and sequences are denoted by lowercase bold letters. Matrices and multidimensional-arrays are denoted by uppercase bold letter. Sets are denoted by uppercase blackboard-bold letters. Latin or Greek letters can be used for any type of object.}
\glsentm{o}{var}{a,\,A,\,\theta}
	{Variable (commonly scalar or function)}
\glsentm{o}{vec}{\vec a,\,\vec\theta}
	{Vector or sequence (commonly column vector)}
\glsentm{o}{mat}{\vec A,\,\vec\Theta}
	{Matrix or multidimensional array}
\glsentm{o}{set}{\set A}
	{Set or multiset}
\glsentm{o}{const}{\const a,\,\const A,\,\uptheta}
	{Constant}
\glsentm{o}{cvec}{\cvec a,\,\boldsymbol{\uptheta}}
	{Vector or sequence constant}
\glsentm{o}{cmat}{\cvec A,\,\cvec\Theta}
	{Matrix or multidimensional array constant}
\glsentm{o}{cset}{\cset A}
	{Set constant}
\glsentm{o}{rvar}{\rvar a,\,\rvar A,\,\rvar\theta}
	{Random variable}
\glsentm{o}{rvec}{\rvec a,\,\rvec\theta}
	{Random vector or sequence}
\glsentm{o}{rmat}{\rvec A,\,\rvec\Theta}
	{Random matrix or multidimensional array}
\glsentm{o}{rset}{\rset A}
	{Random set or multiset}
\glsentm{o}{text}{\text{a},\,\text{riječ}}
	{Textual label not representing an object}

% Konstante
\glsgroup{k}{Constants}{}
\glsentm{k}{emptyset}{\cbr{}}
	{Enpty set}
\glsentm{k}{e}{\const e}
	{The number that satisfies $\od{}{x}\const e^x=\const e^x$}
\glsentm{k}{nulvek}{\cvec 0}
	{Null-vector}
\glsentm{k}{kanvek}{\cvec e_i}
	{$i$-th canonical basis vector}
\glsentm{k}{jedvek}{\cvec 1}
	{The sum of all canonical basis vectors}
\glsentm{k}{mati}{\cvec I,\,\cvec I_n}
	{Identity matrix (with $n$ rows/columns)}
\glsentm{k}{cset}{\N,\Z,\R,\C}
	{A standard set}
\glsentm{k}{Rpos}{\R_{\geq 0},\,\R_{> 0}}
	{The set of all non-negative/positive real numbers}

% Skupovi i nizovi
\glsgroup{sn}{Defining sets and arrays}{}
\glsentm{sn}{range}{a\bidot b}
	{Shorthand notation for $a,..,b$}
\glsentm{sn}{setrange}{\cbr{a\bidot b}}
	{A subset of integers from $a$ to $b$}
\glsentm{sn}{setdefset}{\cbr{f(a)\colon P(a)},\, \cbr{f(a)}_{P(a)}}
	{A set with elements defined by a function $f$ and a predicate $P$}
\glsentm{sn}{setdefsetimp}{\cbr{f(a)}_{a}}
	{A set with elements defined by a function $f$ and variables $a$ from an implicitly defined set}
\glsentm{sn}{setdefn}{\cbr{a_1\bidot a_n},\,\cbr{a_i}_{i=1\bidot n}}
	{A set with $n$ elements}
\glsentm{sn}{rowvec}{\sbr{x_1,\bidot,x_n}}
	{A row vector}
\glsentm{sn}{ndarrdef}{\sbr{a_i}_{i}, \sbr{a_{i,j}}_{i,j}, \sbr{a_{i,j,k}}_{i,j,k}}
	{A multidimensional array with an implicit or undefined number of elements}
\glsentm{sn}{intco}{\intco{a,b}}
	{A semi-closed interval}
%\glsentm{sn}{colvec}{\del{x_1,\bidot,x_n}}
%	{$n$-torka}

% Donji i gornji indeks
\glsgroup{i}{Subscript and superscript}
{U donjem i gornjem indeksu oznake mogu biti oznake drugih matematičkih objekata ili slova ili riječi koje ne predstavljaju matematičke objekte. Redni brojevi elemenata vektora ili višedimenzionalnih nizova se, ako nije određeno drugačije, pišu u donjem indeksu oznake vektora u uglatim zagradama. Npr. $i$-ti element vektora $\vec a=\sbr{a_1,.., a_n}^\tp$ je $\vec a_\ind{i}=a_i$. Indeksi kod $n$-dimenzionalnih nizova mogu biti i vektori iz $\N^{n}$, ili kombinacije vektora manje dimenzije sa skalarima.}
\glsentm{i}{gdindeks}{a_\text{d}^\text{g}}
	{Varijabla s oznakama u donjem i gornjem indeksu}
\glsentm{i}{vecelem}{\vec{a}_\ind{i}}
	{$i$-ti element vektora $\vec{a}$}
\glsentm{i}{subvec}{\vec{a}_\ind{i_1:i_2}}
	{Vektor kojeg čine elementi $\vec{a}_\ind{i_1}, \vec{a}_\ind{i_1+1},.., \vec{a}_{\sbr{i_2}}$}
\glsentm{i}{subvecsk}{\vec{a}_\ind{(i_1\bidot i_n)}}
	{Vektor kojeg čine elementi $\vec{a}_\ind{i_1}, \vec{a}_\ind{i_2},.., \vec{a}_{\sbr{i_n}}$}
\glsentm{i}{matelem}{\vec{A}_\ind{i,j}}
	{Element $i,j$ matrice $\vec A$}
\glsentm{i}{matrow}{\vec{A}_\ind{i,:}}
	{$i$-ti redak matrice $\vec A$}
\glsentm{i}{asubmat}{\vec{A}_\ind{:,i_1:i_2,j}}
	{2-D odsječak 3-D niza $\vec A$}
\glsentm{i}{aet}{\vec{A}_\ind{\vec i}}
	{Element $\vec{A}_\ind{\vec i_\ind{1},\bidot,\vec i_\ind{n}}$ $n$-D niza}
\glsentm{i}{ast}{\vec{A}_\ind{\vec i_1:\vec i_2}}
	{Podniz $\vec{A}_\ind{{\vec i_1}_\ind{1}:{\vec i_2}_\ind{1},\bidot,{\vec i_1}_\ind{n}:{\vec i_2}_\ind{n}}$ $n$-D niza}
\glsentm{i}{astp}{\vec{A}_\ind{\vec i_1:\vec i_2;:}}
	{Podniz $\vec{A}_\ind{{\vec i_1}_\ind{1}:{\vec i_2}_\ind{1},\bidot,{\vec i_1}_\ind{n-1}:{\vec i_2}_\ind{n-1},:}$ $n$-D niza}

% Operacije linearne algebre i operacije s nizovima
\glsgroup{l}{Operacije linearne algebre i operacije s nizovima} {} 
\glsentm{l}{scalprod}{\braket{\vec a}{\vec b},\,\vec{a}^\tp\vec{b}}
	{Skalarni produkt}
\glsentm{l}{outprod}{\vec{a}\vec{b}^\tp}
	{Vanjski produkt}
\glsentm{l}{hadprod}{\vec a \odot \vec b}
	{Umnožak po elementima; Hadamardov produkt}
\glsentm{l}{haddiv}{\vec a \oslash \vec b}
	{Dijeljenje po elementima}
\glsentm{l}{hadpow}{\vec a^{\odot b}}
	{Potenciranje po elementima}
\glsentm{l}{matmul}{\vec A \vec B}
	{Matrično množenje}
\glsentm{l}{matinv}{\vec A^{-1}}
	{Inverz matrice}
\glsentm{l}{transp}{\vec A^\tp}
	{Transponiranje}
\glsentm{l}{diag}{\diag\del{\vec{a}}}
	{Dijagonalna matrica kojoj dijagonalu čini vektor $\vec a$}
\glsentm{l}{det}{\det(\vec{A})}
	{Determinanta matrice $\vec A$}
\glsentm{l}{vecl2norm}{\enVert{\vec a}_2}
	{$\const L^2$-norma vektora $\vec a$}
\glsentm{l}{vecnorm}{\enVert{\vec a}_p}
	{$\const L^p$-norma vektora $\vec a$}
\glsentm{l}{matnorm}{\enVert{\vec A}_p}
	{Matrična $\const L^p$-norma matrice $\vec A$}
\glsentm{l}{frobnorm}{\enVert{\vec A}_\text{F}}
	{Frobeniusova norma matrice $\vec A$}
%\glsentm{l}{func}{f(\vec a)}
%	{Ako $f$ nije drugačije definirana i inače označava funkciju $\R\to\R$, onda se primjenjuje po elementima}
\glsentm{l}{conc}{\vec a\concat\vec b}
	{Konkatenacija vektora (stupaca) $\vec a\in\R^n$ i $\vec b\in\R^m$ u vektor iz $\R^{n+m}$}
\glsentm{l}{conc1}{\vec A\concat\vec B}
	{Konkatenacija nizova po prvoj dimenziji}
%\glsentm{l}{dconc}{\vec A\concat'\vec B}
%	{Konkatenacija nizova po zadnjoj dimenziji}
\glsentm{l}{vec}{\vecfunc(\vec A)}
	{Funkcija koja preslikava niz iz $\R^{d_1\times\dots\times d_n}$ u $\R^{d_1\dots d_n}$}
\glsentm{l}{vdim}{\dim(\vec a)}
	{Dimenzija vektora}
\glsentm{l}{dim}{\dim(\vec A)}
	{Vektor dimenzija niza; $\sbr{d_1,\bidot,d_n}$ za $\vec A\in\R^{d_1\times\dots\times d_n}$}

% Diferencijalni račun
\glsgroup{d}{Diferencijalni račun}{}
\glsentm{d}{od}{\tod{y}{x},\,\tod{}{x}f(x)}
	{Derivacija $y=f(x)$ po $x$}
\glsentm{d}{pd}{\tpd{y}{x},\,\tpd{}{x}f(x)}			
	{Parcijalna derivacija $y=f(x)$ po $x$}
\glsentm{d}{grad}{\nabla_{\vec x}{y},\,\nabla_{\vec x}{f(x)},\,\del{\tpd{y}{\vec x}}^\tp} 	
	{Gradijent $y=f(\vec x)$ po $\vec x$}
\glsentm{d}{gradmat}{\nabla_{\vec X}{y},\,\nabla_{\vec X}{f(x)}}	
	{Gradijent $y=f(\vec x)$ po $\vec X$}
\glsentm{d}{hess}{\tfrac{\partial^2y}{\partial\vec x\partial\vec x^\tp},\,\vec H_{f}(\vec x),\,\vec H}
	{Hessijan iz $\R^{n\times n}$ za $\funcdef{f}{\R^n}{\R}$ i $y=f(\vec x)$}
\glsentm{d}{jacobi}{\tpd{\vec y}{\vec x},\,\vec J_{f}(\vec x),\,\vec J}
	{Jakobijeva matrica iz $\R^{m\times n}$ za $\funcdef{f}{\R^n}{\R^m}$ i $\vec y=f(\vec x)$}
\glsentm{d}{int}{\int_{\set A}f(x)\dif x,\,\int_{x\in\set A}f(x)}
	{Određeni integral funkcije $f(x)$ po $x\in\set A$}
\glsentm{d}{int2}{\int f(x)\dif x,\,\int_x f(x)} 
	{Određeni integral funkcije $f(x)$ po $x\in\set A$, gdje je $\set A$ implicitan}

% Teorija vjerojatnosti
\glsgroup{tv}{Teorija vjerojatnosti}
{Svakoj slučajnoj varijabli $\rvar a$ jednoznačno je dodijeljena jedna razdioba $\p(\rvar a)$ (ili $\P(\rvar a)$) i funkcija gustoće vjerojatnosti (koja može biti poopćena funkcija) $p_{\rvar a}(a)=\p(\rvar a=a)$. $\P(A)$ označava vjerojatnost događaja $A$, a $P_{\rvar a}$ funkciju vjerojatnosti slučajne varijable $\rvar a$. Mogući su i kraći zapisi $\p(a)$ i $\P(a)$, gdje se po slovu koje označava vrijednost pretpostavlja slučajna varijabla označena istim slovom bez serifa. Mogu se koristiti i druge oznake za funkciju vjerojatnosti ili funkciju gustoće vjerojatnosti.}
%TODO move to new page if too high
\glsentm{tv}{rvarcond}{(\rvar a\mid\rvar b= b),\,(\rvar a\mid b)}{Uvjetna slučajna varijabla}
\glsentm{tv}{rvarjoint}{(\rvar a,\rvar b)}{Združena slučajna varijabla}
\glsentm{tv}{indep}{\rvar a\perp\rvar b}{\textit{Slučajne varijable $\rvar a$ i $\rvar b$ su nezavisne}}
\glsentm{tv}{dep}{\rvar a\not\perp\rvar b}{\textit{Slučajne varijable $\rvar a$ i $\rvar b$ su zavisne}}
\glsentm{tv}{condindep}{\rvar a\perp\rvar b\mid\rvar c}{\textit{Slučajne varijable $\rvar a$ i $\rvar b$ su uvjetno nezavisne uz poznat ishod slučajne varijable $\rvar c$}}
\glsentm{tv}{conddep}{\rvar a\not\perp\rvar b\mid\rvar c}{\textit{Slučajne varijable $\rvar a$ i $\rvar b$ su uvjetno zavisne uz poznat ishod slučajne varijable $\rvar c$}}
\glsentm{tv}{distr}{p,\,q}{Razdioba ili funkcija gustoće vjerojatnosti}
\glsentm{tv}{event}{\set A}{Događaj}
\glsentm{tv}{eventpred}{\cbr{R(\rvar a)}} {Događaj definiran predikatorm slučajne varijable $\rvar a$}
\glsentm{tv}{prob}{\P(\cbr{R(\rvar a)}),\,\P(R(\rvar a))} {Vjerojatnost događaja $\cbr{R(\rvar a)}$}
\glsentm{tv}{distrrvar}{\P(\rvar a),\,\p(\rvar a),\,\mathcal{D}} {Razdioba slučajne varijable $\rvar a$; $\P$ ako je $\rvar a$ diskretna slučajna varijabla, a $\p$ ako nije ili ako se ne zna}
\glsentm{tv}{probel}{\P(\rvar a= a),\,P_{\rvar a}(a),\,\P(a)} {Vjerojatnost događaja $\cbr{\rvar a=a}$}
\glsentm{tv}{probdens}{\p(\rvar a= a),\,p_{\rvar a}(a),\,\p(a)} {Gustoća vjerojatnosti događaja $\cbr{\rvar a=a}$}
\glsentm{tv}{pdfcond}{p_{\rvar a\mid b}(a),\,\p(a\mid b)} {Gustoća vjerojatnosti događaja $\cbr{\rvar a=a\mid\rvar b=b}$}
\glsentm{tv}{pdfjoint}{p_{\rvar a,\rvar b}(a,b),\,\p(a,b)} {Gustoća vjerojatnosti događaja $\cbr{\rvar a=a,\rvar b=b}$}
\glsentm{tv}{hasdistrib}{\rvar a \sim q,\, \p(\rvar a)=q} {\textit{Slučajna varijabla $\rvar a$ ima razdiobu $q$}}
\glsentm{tv}{hasdistribset}{\rvar a \sim \set A} 	{\textit{Slučajna varijabla $\rvar a$ ima takvu razdiobu da svi elementi (multi)skupa $\set A$ imaju vjerojatnost proporcionalnu višestrukosti ($\frac{1}{\envert{\set A}}$ za običan skup)}}
\glsentm{tv}{fromdistrib}{a\sim q} {\textit{$a$ se izvlači iz razidiobe $q$}}
\glsentm{tv}{fromrvar}{a\sim \rvar a,\,a\sim \p(\rvar a)} {\textit{$a$ se izvlači iz razidobe $\p(\rvar a)$}}
\glsentm{tv}{E}{\E_{a\sim\rvar a} f(a),\,\E_{\rvar a} f(a)} {Očekivanje funkcije slučajne varijable $\rvar a$}
\glsentm{tv}{D}{\D_{a\sim\rvar a} f(a),\,\D_{\rvar a} f(a)} {Disperzija (varijanca) funkcije slučajne varijable $\rvar a$}
\glsentm{tv}{Cov}{\Cov(\rvar a,\rvar b)}		{Kovarijanca}
\glsentm{tv}{Gauss}{\mathcal{N}(\mu, \sigma^2)} {Normalna razdioba s učekivanjem $\mu$ i varijancom $\sigma^2$}
\glsentm{tv}{unif}{\mathcal{U}(\set A)}
	{Uniformna razdioba nad skupom $\set A$}

% Information theory
\glsgroup{ti}{Information theory}{}
\glsentm{ti}{I}{\I(\set A)}
	{Information content of event $\set A$}
\glsentm{ti}{entropy}{\H(\rvar a)}
	{Entropy}
\glsentm{ti}{diffent}{\h(\rvar a)}
	{Differnetial entropy}
\glsentm{ti}{mutinf}{\I(\rvar a;\rvar b)}
	{Mutual information}
\glsentm{ti}{condent}{\H(\rvar a\mid\rvar b)}
	{Conditional entropy}
\glsentm{ti}{crossent}{\H_{\rvar b}(\rvar a)}
	{Cross entropy}
\glsentm{ti}{relent}{\Dist_{\rvar b}(\rvar a)}
	{Relative entropy (Kullback-Leibler divergence)}

% Grafovi
\glsgroup{g}{Grafovi}{}
\glsentm{g}{pa}{\pa_G(a)}
	{Skup čvorova roditelja čvora $a$ u grafu $G$}
\glsentm{g}{ch}{\ch_G(a)}
	{Skup čvorova djece čvora $a$ u grafu $G$}
\glsentm{g}{pred}{\pred_G(a)}
	{Skup čvorova prethodnika čvora $a$ u grafu $G$}
\glsentm{g}{succ}{\succ_G(a)}
	{Skup čvorova nasljednika čvora $a$ u grafu $G$}

% Ostalo
\glsgroup{f}{Other}{}
\glsentm{f}{funcset}{\set A\to\set B}
	{A set of functions with domain $\set A$ and codomain $\set B$}
%\glsentm{f}{func}{\funcdef{f}{\set A}{\set B}}
%	{Funkcija s domenom $\set A$ i kodomenom $\set B$}
%\glsentm{f}{fusnc}{f\in(\set A\to\set B)}
%	{Funkcija s domenom $\set A$ i kodomenom $\set B$}
\glsentm{f}{funcdefs}{f\in(\set A\to\set B)}
	{Function definition; a function mapping from its domain $\set A$ to its codomain $\set B$}
\glsentm{f}{funcdef}{x\mapsto f(x)}
	{Function definition; a function mapping $x$ from its domain to $f(x)$ in its codomain}
\glsentm{f}{fadd}{f+g}
	{Sum of functions}
\glsentm{f}{fmul}{fg}
	{Product of functions}
\glsentm{f}{conv}{f*g}
	{Convolution of functions}
\glsentm{f}{fscalprod}{\braket{f}{g}}
	{Scalar product of functions}
\glsentm{f}{card}{\envert{\set A}}
	{Set cardinality}
\glsentm{f}{dirac}{\dirac\del{\cdot}}
	{Dirac delta}
\glsentm{f}{doublebracket}{\enbbracket{\cdot}}
	{Iverson bracket; $\enbbracket{P}=\begin{cases} 1, & P \equiv \top \\ 0, & P \equiv \bot \end{cases}$}
\glsentm{f}{image}{f\sbr{\set A}}
	{Image of $\set A$ (subset of the domain) under $f$; $f\sbr{\set A} \coloneqq \cbr{f(a)\mid a\in\set A}$}
\glsentm{f}{invimage}{f^{-1}\sbr{\set B}}
	{Inverse image (preimage) of $\set B$ (subset of the codomain) under $f$; $f^{-1}\sbr{\set B} \coloneqq \cbr{a\mid f(a)\in\set B}$}


\begin{document}

\maketitle

\tableofcontents
%\listoffigures
%\listoftables

\newpage

\begingroup
\onehalfspacing
\printunsrtglossary[type=symbols,style=supergroup,title={Notation}]
\endgroup




\section{Probability}

%http://www.workinginuncertainty.co.uk/probtheory_notation.shtml

\textbf{Probability of an event} $\P(E)$

The distribution of a random variable $\rvar x$ is denoted $\p_\rvar{x}$. If $\rvar x$ is known to be discrete, its distribution can be denoted with $\P_\rvar{x}$. 
\begin{align}
    \P_\rvar{x}(x) &= \P(\rvar x = x) \text.
\end{align}
\begin{align}
    \p_\rvar{x}(x) &= \lim_{\epsilon\to0}\frac{\P\del{\rvar x\in B_\epsilon(x)}}{\int_{x'\in B_\epsilon(x)}\dif x'} \text.
\end{align}
\begin{align}
    \P_\rvar{x}(x\in A) = \int_{x\in A}\dif \p_\rvar{x}(x) \text.
\end{align}

\subsection{Functions of random variables}

For any function $f\in \set A\to\set B$, we will use the same symbol to denote an equivalent function that maps random variables taking values in $\set A$ to random variables taking values in $\set B$:
\begin{align}
    \P(f(\rvar a)\in \set B_1) \coloneqq \P\del{\rvar a\in f^{-1}\sbr{\set B_1}} \text,
\end{align}
where $f^{-1}\sbr{\set B_1} \coloneqq \cbr{a\mid f(a)\in \set B_1}$ is what we call the preimage of $\set B_1\subseteq\set B$ under $f$.

\textbf{Conditional expectation}:
\begin{align}
    \E\del{\rvar x\mid\rvar y} = \del{y\mapsto \E\del{\rvar x\mid y}}(\rvar{y}) \text.
\end{align}
The conditional expectation $E\del{\rvar x\mid\rvar y}$ is a function of the random variable $\rvar y$ and, thus, is a random variable as well. 

\section{Statistics}

\subsection{Monte Carlo integration}

Monte Carlo integration (approximation) is a method of approximating integrals that can be expressed as expectations of some random variables.

Let $u\in \set A\to \R$ be a function. Let $I\coloneqq\int u(x)\dif x$ be an integral that is hard to compute. If we express $u$ as the product of a function $f$ and a probability density function (distribution) $p$, $u(x) = f(x)p(x)$, the integral $I$ can be expressed as the expectation of $f(\rvar x)$, where $\rvar x$ is distributed according to $p$:
\begin{align}
    I &= \int f(x)\dif x = \int f(x)p(x)\dif x = \E_{x\sim p}(f(x)) \text .
\end{align}
This expectation can be approximated with the following estimator:
\begin{align} 
    \rvar{\hat I}_n \coloneqq \frac{1}{n}\sum_{i=1\bidot n} f(\rvar x_i) \text{,}
\end{align}
where $\rvar x_i\sim p$. The estimator $\rvar{\hat I}_n$ is unbiased if $\rvar x_i$ are independent and it is valid if variances of $u(\rvar x_i)$ are bounded.

\subsection{Rejection sampling}

\subsection{Importance sampling}
$I\coloneqq\int_{x\in \set B}f(x)\dif x$

\section{Information theory}


\subsection{Information-theoretic measures}

functionals

\subsubsection{Basic concepts}

\textbf{Information content} of an event -- optimal message length for the event $\event{\rvar x=x}$:
\begin{align}
\I(\rvar x=x) = -\ln\P(x) \text.
\end{align}

\textbf{Entropy} (Shannon entropy) of a random variable (or a distribution) -- expected message length for optimally encoded elementary events of $\rvar x$:
\begin{align}
    \H(\rvar x) &= \E_{\rvar x}\I(\rvar x=x) = -\E\ln\P(\rvar x) \text.
\end{align}
The same formula applies for joint distributions, e.g.\ for the distribution $\P(\rvar x,\rvar y)$, we denote entropy (also called \textbf{joint entropy}) by $\H(\rvar x,\rvar y)$. Entropy is a common measure of uncertainty.

\textbf{Cross entropy} -- expected message length if the optimal code for $\P_\rvar{y}$ is used, but $\P_\rvar{x}$ is sampled.
\begin{align}
    \H_\rvar{y}(\rvar x) = \E_{\rvar x}\I(\rvar y=x) = -\E_{\rvar x}\ln\P(\rvar y=x) \text.
\end{align}

\textbf{Relative entropy} (Kullback–Leibler (KL) divergence) -- difference of cross entropy and entropy, measures how much $\P_\rvar{y}$ differs from $\P_\rvar{x}$:
\begin{align}
    \Dist_\rvar{y}(\rvar x) = \H_\rvar{y}(\rvar x) - \H(\rvar x) = \E_\rvar{x}\del{\I(\rvar y=x)-\I(\rvar x=x)} = \E_\rvar{x}\ln\frac{\P(x)}{\P(\rvar y=x)} \text.
\end{align}

\textbf{Mutual information} of random variables is the expectation of how much knowing the outcome of one of them gives information (or reduces the uncertainty) about the other:
\begin{align}
    \I(\rvar x; \rvar y) = \H(\rvar x) - \H(\rvar x\mid\rvar y) = \H(\rvar y) - \H(\rvar y\mid\rvar x) = \H(\rvar x) +\H(\rvar y) - \H(\rvar x, \rvar y) \text.
\end{align}
If there is a common condition, we can use $\I\del{\rvar x; \rvar y\mid z}$ as a shorter notation for $\I\del{(\rvar x\mid z); (\rvar y\mid z)}$. If we want to express mutual information between e.g.\ $\rvar x$ and $\rvar y\mid z$, we do it like this: $\I(\rvar x;(\rvar y\mid z))$, without ambiguity.

Mutual information can also be expressed as relative entropy:
\begin{align}
    \I(\rvar x; \rvar y) = \Dist_{\P_{\rvar x}\P_\rvar{y}}(\P_{\rvar x,\rvar y}) \\
    \I(\rvar x; \rvar y) = \Dist_{\P[\rvar x]\P[\rvar y]}(\P[\rvar x,\rvar y]) \\
    \I(\rvar x; \rvar y) = \Dist_{\P(\rvar x)\P(\rvar y)}(\P(\rvar x,\rvar y)) \\
    \I(\rvar x; \rvar y) = \Dist_{[\rvar x][\rvar y]}([\rvar x,\rvar y]) \text.
\end{align}

\subsubsection{Conditional measures}

Conditional counterparts of the information-theoretic measures have random variables in the condition-part of the expression that represents the argument of a measure. e.g.\ \textbf{Conditional entropy} is defined like this:
\begin{align}
    \H(\rvar x\mid\rvar y) &= \E_{\rvar y}\H(\rvar x\mid y) \text.
\end{align}
Similarly, conditional cross-entropy can be defined like this:
\begin{align}
    \H_\rvar{y}(\rvar x\mid\rvar z) &= \E_\rvar{z}\H_\rvar{y}(\rvar x\mid z) \text,
\end{align}
conditional mutual information like this:
\begin{align}
    \I(\rvar x; \rvar y\mid \rvar z) = \E_\rvar{z}\I\del{\rvar x; \rvar y \mid z} \text.
\end{align}

\subsubsection{Differential counterparts}

\subsubsection{Information theory and measure theory}

\url{https://en.wikipedia.org/wiki/Information_theory_and_measure_theory}

\subsection{Kolmogorov complexity}

\subsection{Minimum description length}


\section{Machine learning}

\paragraph{An Occam's razor thought.} If the model (hypothesis search space) is simpler (smaller), we are more likely to find the correct hypothesis. If the correct hypothesis is complex, there will be more hypotheses consistent with the data and we are less likely to find the correct one anyway.


\section{Uncertainty in machine learning}

\subsection{Expressing uncertainty}

The basic and most complete way to express uncertainty are probability distributions. From a probability distribution, other uncertainty measures can be derived. Some common ones are the distribution of a derived random variable (a random variable which is a function of the original one), a parameter or a property of the distribution (e.g.\ the probability of the most certain value of the random variable or the entropy of the distribution).


\subsection{Epistemic and aleatory uncertainty}

%\paragraph{Što je epistemička nesigurnost?} 
\emph{Epistemička nesigurnost} (nesigurnost modela) je nesigurnost u model ili parametre. Ona se može smanjiti uz više podataka/informacija. \emph{Epistemička nesigurnost predikcije} dolazi od nesigurnosti u model/parametre.

%\paragraph{Kad možemo izraziti epistemičku nesigurnost?}
Kad parametre modela procjenjujemo točkasto, nemamo aposteriornu razdiobu parametara i ne znamo kakva je epistemička nesigurnost (ne možemo ju izraziti), ali ju možemo smanjiti uz više podataka. Kod bayesovske procjene parametara ili kod ansabla možemo procijeniti epistemičku nesigurnost.

%\paragraph{Što je aleatorna nesigurnost?}
\emph{Aleatorna nesigurnost} (predikcije) je nesigurnost koja dolazi od višeznačnosti podataka i ograničenja modela. Aleatorna nesigurnost se ne može smanjiti uz više podataka, ali bi se mogla smanjiti uz bolje podatke, tj. podatke koji imaju značajke koje sadrže više korisnih informacija, ili model koji pronalazi bolje značajke.

Kod diskriminativnog modela izlazna razdioba $p(y\mid x, \theta_\text{MAP})$ izražava aleatornu nesigurnost.

\paragraph{Je li ukupna nesigurnost zbroj epistemičke i aleatorne?}
Mislim da procjena ukupne nesigurnosti ovisi o tome koliko je dobro procijenjena epistemička nesigurnost. Što je lošija procjena aposteriorne razdiobe parametara, to je procjena epistemičke nesigurnosti lošija.


\subsection{Nesigurnost i izvanrazdiobni primjeri}

Neka je $D_{\text{train}}$ razdioba iz koje su došli primjeri za učenje. Diskrimanativni model uči funkciju $p(y\mid x)$. Ako je gustoća vjerojatnosti $D_{\text{train}}(x)$ jako mala (ili $0$), moguće je da nije bilo sličnih primjera u skupu za učenje i model može dati bilo kakvu predikciju za taj primjer. Takve primjeri su \emph{izvanrazdiobni primjeri}.

Ipak, pokazano je se da se (kod nekih modela) izvanrazdiobni primjeri često mogu dosta dobro prepoznavati na temelju izlazne razdiobe modela s točkasto procijenjenim parametrima. 


\subsection{Successfulness of epistemic uncertainty estimation with bayesian inference approximation}



\section{Adversarial examples and generalization}

Even for models that perform similar to humans on testing data, it has been shown that, by perturbing input examples even inperceptibly for humans, the models can be made to significantly change their predictions, i.e.\ make confident wrong predictions \citep{Szegedy:2013:IPNN, Goodfellow:2014:EHAE}. Such perturbed input examples are called \textbf{adversarial examples}. The existence of adversarial examples indicates that such models are probably performing well for somewhat wrong reasons, without actually \textit{understanding data}.

\subsection{Defining and finding adversarial examples}

Let $\set X$ be the input space, and $d\in(\set X\times\set X\to \R^+)$ a \textbf{distance function} that can be used to define similarity between inputs. For each example $\vec x$ we can also define its \textbf{neighbourhood} as $B_{\epsilon} = \cbr{\vec x'\colon d(\vec x', \vec x) \leq \epsilon}$, $\epsilon$ being the maximum distance from the example.

Ideally, the neighbourhood of an example $\vec x$ should be the set of \textit{perceptually similar} examples that all belong to the same class as $\vec x$ (their true class may be at most ambiguous), but it is hard to define such a neighbourhood. A practical and common way of defining the neighbourhood function (for images) is to let the distance function $d$ be a $L^p$ distance where $p$ is usually $\infty$ or $2$. Note that if an example is very near the true class boundary, such a neighbourhood may contain examples actually belonging to another class. 

Finding an adversarial example can be defined as an optimization problem of maximizing the loss with respect to the input with the constraint that the input is in the neighbourhood $B_\epsilon(\vec x)$:
\begin{align}
    \tilde{\vec x} = \argmax_{\vec x'\in B_\epsilon(\vec x)} L(y, h(\vec x')) , \label{eq:untargeted-loss-attack}
\end{align}
where $y$ is either the true label or the predicted label. Let $\hat{h}(\vec x) = \argmax_{y'} h(\vec x')_\ind{y'}$ denote the function that assigns the label with the highest probability to an input.
An objective can also be to find the $\tilde{\vec x}$ closest to $\vec x$ such that the classifier misclassifies it \citep{Moosavi-Dezfooli:2016:DFSAMFDNN}:
\begin{align}
    \tilde{\vec x} = \argmin_{\vec x'\colon \vec x'\in B_\epsilon(\vec x) \land \hat{h}(\vec x) \neq y} d(\vec x', \vec x). \label{eq:untargeted-closest-attack}
\end{align}

The described objectives, where it only matters that the adversarial example is misclassified, are objectives for \textbf{untargeted adversarial attacks}. There are also \textbf{targeted adversarial attacks}, where the objective is to create an adversarial example such that the model classifies it as some desired class. Targeted attack objectives corresponding to equations \eqref{eq:untargeted-loss-attack} and \eqref{eq:untargeted-closest-attack} are:
\begin{align}
    \tilde{\vec x} = \argmin_{\vec x'\in B_\epsilon(\vec x)} L(y_\text{t}, h(\vec x')) \label{eq:targeted-loss-attack}
\end{align}
and
\begin{align}
    \tilde{\vec x} = \argmin_{\vec x'\colon \vec x'\in B_\epsilon(\vec x) \land \hat{h}(\vec x) = y_\text{t}} d(\vec x', \vec x), \label{eq:targeted-closest-attack}
\end{align}
where $y_\text{t}$ denotes the adversarial target label.

Adversarial examples can also be generated without knowledge of the true label. Such adversarial examples are called \textbf{virtual adversarial examples}.
\cite{Miyato:2017:VATRMSSSL} propose the following objective for adversarial training:
\begin{align}
    \tilde{\vec x} = \argmin_{\vec x'\in B_\epsilon(\vec x)} D((\rvec y\mid \rvec x, \rvec\theta), (\rvec y\mid \rvec x = \vec x', \rvec\theta)),  
\end{align}
where $D$ is some non-negative function that represents distance between distributions. Other (untargeted) attacks can also produce virtual adversarial examples by using the predicted label $\hat{h}(\vec x)$ instead of the true label in the loss.

\subsubsection{Transferability}

Common (naturally trained) CV models (algorithms) are biased similarly with respect to having adversarial examples -- they are nonrobust in similar ways. 

Can this bias be easily overcome? Unfortunately, there seems not to be much evidence indicating this.

Overdependence on semantically low-level features.

\subsubsection{Adversarial training}

\cite{Kurakin:2016:AMLS} note that by using the true label in the loss in untargeted attacks ($y$ in equation \eqref{eq:untargeted-loss-attack}) can cause

\subsubsection{Distance metrics for images}

Usually, an $L^p$ distance ($d(\vec x', \vec x)=\lVert\tilde{\vec x}-\vec x\rVert_p$) is used as a distance metric for adversarial examples.

\paragraph{Scale-invariant norms.}
Let $\vec x$ denote some image (or perturbation) and $\vec x_\lambda$ the same image with dimensions scaled by $\lambda$. $\vec x_\lambda$ has a greater norm because it contains $\lambda^2$ the number of pixels of the original image and every pixel is approximately effectively repeated $\lambda^2$ times. In terms of $\enVert{\vec x}_p$, its norm can be approximated like this:
\begin{align}
    \enVert{\vec x_\lambda}_p 
    &= \del{\sum_{u\in\cbr{0\bidot\lambda H}}\sum_{v\in\cbr{0\bidot\lambda W}} \envert{{\vec x_\lambda}_\ind{u, v}}^p}^{\frac{1}{p}} \\
    &\approx \del{\sum_{u\in\cbr{0\bidot H}}\sum_{v\in\cbr{0\bidot W}} \lambda^2 \envert{{\vec x_\lambda}_\ind{u, v}}^p}^{\frac{1}{p}} \\
    &= \del{\lambda^2\sum_{u\in\cbr{0\bidot H}}\sum_{v\in\cbr{0\bidot W}} \envert{{\vec x_\lambda}_\ind{u, v}}^p}^{\frac{1}{p}} \\
    &= \lambda^\frac{2}{p} \del{\sum_{u\in\cbr{0\bidot H}}\sum_{v\in\cbr{0\bidot W}} \envert{{\vec x_\lambda}_\ind{u, v}}^p}^{\frac{1}{p}} \\
    &\approx \del{\sum_{u\in\cbr{0\bidot H}}\sum_{v\in\cbr{0\bidot W}} \lambda^2 \envert{{\vec x}_\ind{u, v}}^p}^{\frac{1}{p}} \\
    &= \lambda^\frac{2}{p} \del{\sum_{u\in\cbr{0\bidot H}}\sum_{v\in\cbr{0\bidot W}} \envert{{\vec x}_\ind{u, v}}^p}^{\frac{1}{p}} \\
    &= \lambda^\frac{2}{p} \enVert{\vec x}_p \text{.}
\end{align}

If there is a perturbation in $2$ different resolutions, the higher-resolution one will have a greater norm by approximately a factor of $(\lambda^2)^\frac{1}{p}$, especially if the perturbations don't have too high spatial frequencies. Hence, scale invariant equivalents of $L^p$ norms can be defined by dividing the norm by the scale factor. The scale factor can be relative to the scale of an image with area $1$. Then the total number of pixels $n$ can be used as $\lambda^2$. We can define scale-invariant norms like this:
\begin{align}
    \enVert{\vec x}_{\text{s}p} \coloneqq \frac{\enVert{\vec x}_p}{n^\frac{1}{p}}, \label{eq:scale-invariant-norm}
\end{align}
which is the same as the generalized mean \textbf{generalized mean} (also known as \textbf{power mean}), which is usually defined like this:
\begin{align}
    \enVert{\vec x}_{\text{s}p} \coloneqq \frac{\enVert{\vec x}_p}{n^\frac{1}{p}} = \del{\frac{1}{n}\sum_i \envert{x_i}^p}^\frac{1}{p},
\end{align}
where $n$ is the number of pixels in $\vec x$. Such norms could probably enable more informative comparison of norms between different-resolution images and different datasets, and easier hyperparameter choice for adversarial training.
    \mathrm{M}_p(\vec x) \coloneqq \del{\frac{1}{n}\sum_i \vec x_\ind{i}}^{\frac{1}{p}}. \label{eq:generalized-mean}
\end{align}
Such norms could probably be useful for comparison of norms between different-resolution images and different datasets, and hyperparameter choice for adversarial attacks.

Maybe something similar could be done about objects of different scale?

...

\paragraph{Expectation of scale-invariant $p$-norms of high-dimensional uniformly distributed random vectors.}
Let $\rvec x_n = \del{\rvar x_i}_{i\in\cbr{1\bidot n}}$ be a random vector with $n$ independent elements $\rvar x_i\sim\mathcal{U}\del{\intcc{-\epsilon,\epsilon}}$.
Assuming $n$ is very large, its scale-invariant $p$-norm can\footnote{TODO: prove that the approximation is good for large $n$} be approximated with a single sample $\vec x$:
\begin{align*}
	\E \enVert{\rvec x}_{\text{s}p}^p &\approx \enVert{\vec x}_{\text{s}p} = \del{\frac{1}{n}\sum_i \envert{\vec{x}_\ind{i}}^p}^\frac{1}{p} &&\text{(approximation with a single sample)}\\
	&\approx \del{\E\del{\envert{\rvec{x}_\ind{i}}^p}}^\frac{1}{p} &&\text{(IID elements, large $n$)}  \\
	&= \del{\int_{0}^{1}\envert{\rvec{x}_\ind{i}}^p\dif x}^\frac{1}{p} \\	
	&= \del{\frac{1}{p+1}}^\frac{1}{p} = (p+1)^{-\frac{1}{p}} \text{,}
\end{align*}
which is a monotonically increasing functon of $p$.


\begin{align*}
\E \enVert{\rvec x}_{\text{s}p}^p &\approx \enVert{\vec x}_p = \del{\frac{1}{n}\sum_i \envert{\vec{x}_\ind{i}}^p}^\frac{1}{p}
\end{align*}

\paragraph{Local $p$-norm and hiererchical norm.}
Let $\vec k$ denote a non-negative $2$-D kernel with $\enVert{\vec k}_1=1$, e.g. Gaussian cenetered at $(0,0)$. Let $\vec x$ denote an image perturbation and assume that it has a single channel for simplicity.
We can define the local $p$-norm around a pixel $(i, j)$ as
\begin{align}
    \mathrm{LocalNorm}_{p,\vec k}(\vec x)_\ind{i,j} = \del{\envert{\vec x}^p * \vec k}_\ind{i,j}^\frac{1}{p},
\end{align}
where the absolute value and powering operatons are elementwise.

We can then define a bi-level hierarchical $(p_1,p_2)$-norm as $\enVert{\mathrm{LocalNorm}_{p_1,\vec k}(\vec x)}_{p_2}$. This can be generalizad to a multi-level norm $(p_1,..,p_n)$-norm by chaining multiple local norms with potentially different kernels until the last, global, $p_n$-norm.

Why?

%Using e.g a $(1,\infty)$-norm for contraining adversarial perturbations could allow for more flexible perturbations locally, but having local changes not add up to the global norm.

%To make the analysis easier, we can consider $\vec x$ and $\vec x_\lambda$ to be continous functions with delta-peaks at the coordinates of pixels. 

%\begin{align}
%    \enVert{\vec x_\lambda}_p = \del{ \int_{u\in\intcc{0,\lambda h}\cap\N}\int_{v\in\intcc{0,\lambda w}\cap\N} {\vec x_\lambda}\del{u, v}}^{\frac{1}{p}}
%\end{align}

\subsection{Making adversarially robust classifiers}



\section{Generative adversarial networks}

\subsection{Getting the probability of the example from the generator}

first paragraph

case 1) normal generator

case 2) invertible generator


\section{Paper summaries}

\subsection{The Conditional Entropy Bottleneck (Anonymous, 2018)}

URL: \url{https://openreview.net/forum?id=rkVOXhAqY7}.


\section{p}

Neka je odnos između slučajnih varijabli $\rvar x$ i $\rvar y$ definiran funkcijom $f$ koja ishode jedne slučajne varijable deterministički preslikava u ishode druge, što označavamo ovako: $\rvar y = f(\rvar x)$.  Ako su $\rvar x$ i $\rvar y$ diskretne slučajne varijable, onda je razdioba slučajne varijable $\rvar y$ definirana ovako:
\begin{align}
	P_{\rvar y}(y) = \sum_{x\colon f(x)=y} P_\rvar{x}(x) \text{.}
\end{align} 
Ako su $\rvar x$ i $\rvar y$ kontinuirane slučajne varijable s vrijednostima iz $\R$ i $f$ je injektivna, može se pokazati \citep{Elezovic:2007:VSSV} da vrijedi
\begin{align} \label{eq:gustoca-funkcije-sv}
p_{\rvar y}(y) = p_\rvar{x}(x) \envert{\od{x}{y}} \text{.}
\end{align} 
Neka je $C_{\rvar x}(x) := \int_{-\infty}^{x} p_{\rvar x}(x') \dif{x'}$. Vrijednosti iz intervala $\intoo{x, x+\epsilon}$ na kojem je $f$ monotono rastuća preslikavaju se u interval $\intoo{f(x), f(x+\epsilon)}$. Granice su obrnute ako je $f$ monotono padajuća na tom intervalu. Budući da $\P\del{\rvar x\in\intoo{x, x+\epsilon}}=\P\del{\rvar y\in\intoo{f(x), f(x+\epsilon)}}$, vrijedi
\begin{align}
C_{\rvar x}(x+\epsilon)-C_{\rvar x}(x) = 
C_{\rvar y}(f(x+\epsilon))-C_{\rvar y}(f(x)) \text{.}
\end{align}
Ako obje strane jednadžbe dijelimo s $\epsilon$ i pustimo $\epsilon\to0$, 
\begin{align}
	\lim_{\epsilon\to 0}\frac{C_{\rvar x}(x+\epsilon)-C_{\rvar x}(x)}{\epsilon} = \lim_{\epsilon\to 0}\frac{C_{\rvar y}(f(x+\epsilon))-C_{\rvar y}(f(x))}{\epsilon} \text{.}
\end{align}
Redom, prema definiciji derivacije, pravilu derivacije složene funkcije i definiciji funkcija $C_\rvec{x}$ i $C_\rvec{y}$ kao integrala gustoće vjerojatnosti, slijedi:
\begin{align}
\od{}{x}C_{\rvar x}(x) &= \od{}{x}C_{\rvar y}(f(x)) \text{,}\\
\od{}{x}C_{\rvar x}(x) &= \od{}{f(x)}C_{\rvar y}(f(x))\od{}{x}f(x) \text{,}\\
p_{\rvar x}(x) &= p_{\rvar y}(f(x))\od{}{x}f(x) \text{.} \label{eq:gustoca-funkcije-sv-dokaz-rastuci}
\end{align}
Može se pokazati da je za monotono padajuće intervale desna strana jednadžbe~\eqref{eq:gustoca-funkcije-sv-dokaz-rastuci} pomnožena s $-1$, iz čega uz jednadžbu~\eqref{eq:gustoca-funkcije-sv-dokaz-rastuci} slijedi
\begin{align}
p_{\rvar x}(x) &= p_{\rvar y}(y)\envert{\od{y}{x}} \text{,} \label{eq:gustoca-funkcije-sv-dokaz-xy}
\end{align}
gdje je $f(x)$ zamijenjen s $y$. Množenjem toga s $\envert{\od{x}{y}}=\envert{\od{y}{x}}^{-1}$ slijedi jednadžba~\eqref{eq:gustoca-funkcije-sv}. To pravilo se može poopćiti i na vektore. Onda vrijedi \citep{Murphy:2012:MLPP}
\begin{align}
p_{\rvec y}(\vec y)=p_\rvec{x}(\vec x)\envert{\det\pd{\vec x}{\vec y}} \text{.}
\end{align}

Neka je $\rvar z$ zbroj slučajnih varijabli $\rvar x$ i $\rvar y$. Onda vrijedi
\begin{align}
	p_{\rvar z}(z) = \int p_{\rvar x,\rvar y}(x, z-x)\dif x \text{.}
\end{align}
Ako su $\rvar x$ i $\rvar y$ nezavisne, onda to postaje konvolucija:
\begin{align} \label{eq:nezavisne-gustoce-konvolucija}
p_{\rvar z}(z) = \int p_{\rvar x}(x)p_{\rvar y}(z-x)\dif x \eqqcolon (p_{\rvar x}*p_{\rvar y})(z) \text{.}
\end{align}

\subsection{PDF of vector r.v. defined via a function of a vector r.v.}

Let $f \in (\R^n\to\R^m)$ and $\rvec y = f(\rvec x)$. We want to compute the PDF of $\rvar y$, or, equivalently, the distribution $\p(\rvec y)$.
\begin{align}
    \pd{\rvec y}{\rvec x} \in \R^{m\times n}
\end{align}

For easier analysis, let's assume that $m=1$, i.e.\ $\rvec y$ is a scalar, and denote it with $\rvar y$. We want to compute its PDF.


\section{Dense anomaly detection for dense prediction based on reconstruction error}

Pretpostavljamo duboki diskriminativni model $h(\vec x;\vec\theta)$ s parametrima $\vec\theta$ koji ulaz $\vec x$ preslikava u vektor $\vec y$ koji predstavlja izlaznu razdiobu  $\p(\rvar y\mid \vec x, \vec\theta)$.

previsoka sigurnost (postizanje male pogreške na skupu za učenje, kalibracija temperaturnim skaliranjem)

kriva klasifikacija izvanrazdiobnih primjera

neprijateljski primjeri

\citep{Hendrycks:2016:BDMOODE}

\citep{Guo:2017:CMNN}

\citep{Lee:2017:TCCCDOOD}

\citep{Liang:2017:PDOODENN}

Neki pristupi za prepoznavanje anomalija/izvanrazdiobnih primjera (detaljnije opisati i s referencama):
\begin{itemize}
    \item iz predikcije -- očekujemo manju vjerojatnost i veću nesigurnost za izvanrazdioben primjere,
    \item iz neke skrivene reprezentacije -- možemo analizirati razdiobe logita ili nečega drugoga i pomoću toga propoznavati izvanrazdiobne primjere,
    \item eksplicitnim učenjem razlikovanja razdiobe skupa za učenje od neke pozadinske razdiobe,
    \item korištenje generativnog modela za generiranje primjera iz područja male gustoće vjerojatnosti i korištenje njih kao izvanrazdiobnih primjera
    \item korištenjem generativnog modela kod kojeg je moguće izračunati gustoću vjerojatnosti za primjer,
    \item korištenjem rekonstrukcijske pogreške autoenkodera.
\end{itemize}
Neki pristup ise mogu kombinirati.


\subsection{Autoencoders and GAN-s}

.


\subsection{Korištenje rekonstrukcijske pogreške autoenkodera za propoznavanje onoga što model ne zna da ne zna}

Pretpostavljamo duboki diskriminativni model $h(\vec x;\vec\theta)$ s parametrima $\vec\theta$ koji ulaz $\vec x$ preslikava u vektor $\vec y$ koji predstavlja izlaznu razdiobu  $\p(\rvar y\mid \vec x, \vec\theta)$.


\subsubsection{Korištenje autoenkodera za prepoznavanje izvanrazdiobnih primjera}

\cite{Sabokrou:2018:ALOCCND} za otkrivanje anomalija u slici predlažu korištenje autoenkodera (s jako velikom skrivenom reprezentacijom) kojemu se kod učenja kao ulaz daje zašumljena slika. Uz autoenkoder se dodaje diiskriminator koji se uči a razlikuje izlaz autoenkodera od stvarnih primjera za učenje. Kao gubitak se koristi težinski zbroj kvadratne rekonstrukcijske pogreške i suparničkog gubitka. Kao primjeri se koriste mali izrazani dijelovi većih slika. Za prepoznavanje anomalija koristi se izlaz diskriminatora za rekonstruirani primjer.

\cite{Pidhorskyi:2018:GPNDAA} isto predlažu pristup s autoenkoderom i superničkim gubitkom. Kod njih gubitak ima $3$ komponente: (1) suparnički gubitak koji potiče da primjeri za učenje "pokrivaju" cijelu zadanu (Gaussovu) razdiobu skrivene reprezentacije, (2) suparnički gubitak koji potiče da rekonstruirani primjeri budu iz razdiobe skupa za učenje (kao kod \cite{Sabokrou:2018:ALOCCND}) i (3) rekonstrukcijski gubitak. Kao mjera za procjenu je li primjer izvan razdiobe se koristi procjena $\p(\vec z\mid \set D)$ koja ovisi o udaljenosti od "manifolda". Trebam još pručiti kako se točno dobiva.

Pretpostavljamo duboki diskriminativni model $h(\vec x;\vec\theta)$ s parametrima $\vec\theta$ koji ulaz $\vec x$ preslikava u vektor $\vec y$ koji predstavlja izlaznu razdiobu  $\p(\rvar y\mid \vec x, \vec\theta)$. Želimo prepoznavati izvanrazdiobne primjere pomoću autoenkodera.

Neke ideje u vezi autoenkodera:
\begin{itemize}
    \item Koristiti dekoder s heteroskedastičkom \citep{Kendall:2017:WUNBDLCV} nesigurnošću u rekonstrukciju (modelirati $\p(\rvec x\mid \vec z)$) i $-\ln\p(\rvec x\mid \vec z)$ za empirijski gubitak.
    \item Isprobati rekonstrukciju neke skrivene reprezentacije klasifikatora kako bi se u rekonstrukcijskoj pogrešci naglasile značajke bitne za klasifikaciju (semantički bitne). Možemo $h$ rastaviti na dvije funkcije: $h(\vec x) = (f_2\circ f_1)(\vec x)$ pa onda učime autoenkoder rekonstruirati $f_1(\vec x)$. Ako kao kao $f_1$ koristimo bijekciju, možemo vidjeti kako izgleda rekonstrukcija ulaza koja odgovara rekonstruiranoj reprezentaciji.
    \item Isprobati klasifikaciju na temelju skrivene reprezentacije autoenkodera $\vec z$, koristiti i klasifikacijski gubitak za učenje kodera, vidjeti kako izgledaju rekonstrukcije. (Isprobati CEB?)
    \item Minimalna reprezentacija autoenkodera onemogućuje neprijateljske primjere kojima je cilj postići dobru rekonstrukciju anomalije, pogotovo ako pretpostavimo dovoljno dobru funkciju rekonstrukcijske pogreške ili diskriminator.
    \item Je li dobro poticati da skup za učenje pokriva cijelu razdiobu $\p(\rvec z)$? Onda će različite klase biti odmah jedna uz drugu -- malo izmijenimo $\vec z$ i dođemo u područje visoke gustoće za neku drugu klasu. Možda valja učiti razdiobu $\p(\rvec z\mid \set D)$ i znati koja su područja niže gustoće (margine) (kako?).
    \item Dodati šum na ulaz autoenkodera. Možda bi valjalo nešta između gaussovog šuma i "rupa" za popunjavanje.
\end{itemize}

Osnovni model koji bih htio isprobati (na velikim slikama) je ovakav:
\begin{alignat}{3}
    &\vec x \overset{f_1}{\longmapsto} \vec h \overset{f_2}{\longmapsto} \vec y 
    \quad&&\text{(klasifikator)} \text, \\
    &\vec h \overset{e}{\longmapsto} \vec z \overset{d}{\longmapsto} \vec h_\text{r}
    \quad&&\text{(autoenkoder skrivene reprezentacije)} \text.
\end{alignat}
Treba odrediti točan opis modela.

Možemo isprobati i klasifikaciju na temelju rekonstrukcije:
\begin{align}
    &\vec h_\text{r} \overset{f_2}{\longmapsto} \vec y\text.
\end{align}

Možemo isprobati i klasifikaciju na temelju skrivene reprezentacije:
\begin{align}
    &\vec z \overset{f_z}{\longmapsto} \vec y \text,
\end{align}
i istovremeno učenje klasifikacije i rekonstrukcije.

Bilo bi zanimljivo vidjeti kako izgleda rekonstrukcija ulazne slike na temelju izlaza autoenkodera ovisno o tome koji skriveni sloj se kodira:
\begin{align}
    &\vec h_\text{r} \overset{f_1^{-1}}{\longmapsto} \vec x_\text{r}
    \quad \text{(inverz prvog dijela klasifikatora s ulazom $\vec h_\text{r}$)} \text.
\end{align}
Rekonstrukciju ulazne slike možemo dobiti ako koristimo neki model koji je bijektivan, npr. i-RevNet.


\begin{align}
    \min_{h} L_\text{c}(\vec y, \vec y^*) \\
    \min_{d, e} L_\text{r}(\vec h,\vec h_\text{r})
\end{align}

\subsubsection{Što bih još htio isprobati}

Kombinaciju \cite{Lee:2017:TCCCDOOD} i korištenja izvanrazdiobnih primjera u učenju.


\bibliography{bibliography}
\bibliographystyle{fer} 


\end{document}
