\usepackage[symbols,nopostdot,nonumberlist,section]{glossaries-extra}

%\renewcommand{\glossarypreamble}{\footnotesize}

\newglossarystyle{supergroup}{%
	\setglossarystyle{super}%
	\renewcommand*{\glsgroupskip}{}%
	\renewcommand{\glossentry}[2]{%
		\tabularnewline%
		\multicolumn{2}{p{\textwidth}}{%
			\raggedright\glsentryitem{##1}\glstarget{##1}{\glossentryname{##1}}%
		}% 
		\vspace{2mm}%
		\tabularnewline%
	}%
	\renewcommand{\subglossentry}[3]{%
		\glssubentryitem{##2}%
		\glstarget{##2}{\glossentryname{##2}}&%
		\raggedright\glossentrydesc{##2}\glspostdescription\space##3\tabularnewline%
	}%
}
\newcommand{\test}[1]{ \def\tst{#1} \ifx\tst\empty \typeout{optional argument was omitted} \else \typeout{optional argument was given: '#1'} \fi}
\newcommand{\glsgroup}[3]{%
	\newglossaryentry{#1}{type=symbols, name={{\large \textbf{#2}} \def\temp{#3}\ifx\temp\empty\else\vspace{2mm}\newline #3\fi}, description={}}
}
\newcommand{\glsent}[4]{\newglossaryentry{{#1:#2}}{sort={#2},type=symbols,name={#3},description={#4},parent={#1}}}
\newcommand{\glsentm}[4]{\glsent{#1}{#2}{\ensuremath{\displaystyle#3}}{#4}}

%\setglossarypreamble[symbols]{Ovaj odjeljak sadrži popis velikog broja oznaka koje se koriste u ovom radu. Za neke skupine oznaka napisana su kratka objašnjenja koja dodatno pojašnjavaju i opravdavaju neke oznake. Pojmovi koje označavaju neke oznake detaljnije su objašnjeni u poglavlju~\ref{chap:osnovni-pojmovi}.}

% Objekti
\glsgroup{o}{Objects}
{Variables are generally denoted by italic serif letters. Most constants are denoted by upright serif letters. Random variables are underlined. Vectors and sequences are denoted by lowercase bold letters. Matrices and multidimensional-arrays are denoted by uppercase bold letter. Sets are denoted by uppercase blackboard-bold letters. Latin or Greek letters can be used for any type of object.}
\glsentm{o}{var}{a,\,A,\,\theta}
	{Variable (commonly scalar or function)}
\glsentm{o}{vec}{\vec a,\,\vec\theta}
	{Vector or sequence (commonly column vector)}
\glsentm{o}{mat}{\vec A,\,\vec\Theta}
	{Matrix or multidimensional array}
\glsentm{o}{set}{\set A}
	{Set or multiset}
\glsentm{o}{const}{\const a,\,\const A,\,\uptheta}
	{Constant}
\glsentm{o}{cvec}{\cvec a,\,\boldsymbol{\uptheta}}
	{Vector or sequence constant}
\glsentm{o}{cmat}{\cvec A,\,\cvec\Theta}
	{Matrix or multidimensional array constant}
\glsentm{o}{cset}{\cset A}
	{Set constant}
\glsentm{o}{rvar}{\rvar a,\,\rvar A,\,\rvar\theta}
	{Random variable}
\glsentm{o}{rvec}{\rvec a,\,\rvec\theta}
	{Random vector or sequence}
\glsentm{o}{rmat}{\rvec A,\,\rvec\Theta}
	{Random matrix or multidimensional array}
\glsentm{o}{rset}{\rset A}
	{Random set or multiset}
\glsentm{o}{text}{\text{a},\,\text{riječ}}
	{Textual label not representing an object}

% Konstante
\glsgroup{k}{Constants}{}
\glsentm{k}{emptyset}{\cbr{}}
	{Enpty set}
\glsentm{k}{e}{\const e}
	{The number that satisfies $\od{}{x}\const e^x=\const e^x$}
\glsentm{k}{nulvek}{\cvec 0}
	{Null-vector}
\glsentm{k}{kanvek}{\cvec e_i}
	{$i$-th canonical basis vector}
\glsentm{k}{jedvek}{\cvec 1}
	{The sum of all canonical basis vectors}
\glsentm{k}{mati}{\cvec I,\,\cvec I_n}
	{Identity matrix (with $n$ rows/columns)}
\glsentm{k}{cset}{\N,\Z,\R,\C}
	{A standard set}
\glsentm{k}{Rpos}{\R_{\geq 0},\,\R_{> 0}}
	{The set of all non-negative/positive real numbers}

% Skupovi i nizovi
\glsgroup{sn}{Defining sets and arrays}{}
\glsentm{sn}{range}{a\bidot b}
	{Shorthand notation for $a,..,b$}
\glsentm{sn}{setrange}{\cbr{a\bidot b}}
	{A subset of integers from $a$ to $b$}
\glsentm{sn}{setdefset}{\cbr{f(a)\colon P(a)},\, \cbr{f(a)}_{P(a)}}
	{A set with elements defined by a function $f$ and a predicate $P$}
\glsentm{sn}{setdefsetimp}{\cbr{f(a)}_{a}}
	{A set with elements defined by a function $f$ and variables $a$ from an implicitly defined set}
\glsentm{sn}{setdefn}{\cbr{a_1\bidot a_n},\,\cbr{a_i}_{i=1\bidot n}}
	{A set with $n$ elements}
\glsentm{sn}{rowvec}{\sbr{x_1,\bidot,x_n}}
	{A row vector}
\glsentm{sn}{ndarrdef}{\sbr{a_i}_{i}, \sbr{a_{i,j}}_{i,j}, \sbr{a_{i,j,k}}_{i,j,k}}
	{A multidimensional array with an implicit or undefined number of elements}
\glsentm{sn}{intco}{\intco{a,b}}
	{A semi-closed interval}
%\glsentm{sn}{colvec}{\del{x_1,\bidot,x_n}}
%	{$n$-torka}

% Donji i gornji indeks
\glsgroup{i}{Subscript and superscript}
{U donjem i gornjem indeksu oznake mogu biti oznake drugih matematičkih objekata ili slova ili riječi koje ne predstavljaju matematičke objekte. Redni brojevi elemenata vektora ili višedimenzionalnih nizova se, ako nije određeno drugačije, pišu u donjem indeksu oznake vektora u uglatim zagradama. Npr. $i$-ti element vektora $\vec a=\sbr{a_1,.., a_n}^\tp$ je $\vec a_\ind{i}=a_i$. Indeksi kod $n$-dimenzionalnih nizova mogu biti i vektori iz $\N^{n}$, ili kombinacije vektora manje dimenzije sa skalarima.}
\glsentm{i}{gdindeks}{a_\text{d}^\text{g}}
	{Varijabla s oznakama u donjem i gornjem indeksu}
\glsentm{i}{vecelem}{\vec{a}_\ind{i}}
	{$i$-ti element vektora $\vec{a}$}
\glsentm{i}{subvec}{\vec{a}_\ind{i_1:i_2}}
	{Vektor kojeg čine elementi $\vec{a}_\ind{i_1}, \vec{a}_\ind{i_1+1},.., \vec{a}_{\sbr{i_2}}$}
\glsentm{i}{subvecsk}{\vec{a}_\ind{(i_1\bidot i_n)}}
	{Vektor kojeg čine elementi $\vec{a}_\ind{i_1}, \vec{a}_\ind{i_2},.., \vec{a}_{\sbr{i_n}}$}
\glsentm{i}{matelem}{\vec{A}_\ind{i,j}}
	{Element $i,j$ matrice $\vec A$}
\glsentm{i}{matrow}{\vec{A}_\ind{i,:}}
	{$i$-ti redak matrice $\vec A$}
\glsentm{i}{asubmat}{\vec{A}_\ind{:,i_1:i_2,j}}
	{2-D odsječak 3-D niza $\vec A$}
\glsentm{i}{aet}{\vec{A}_\ind{\vec i}}
	{Element $\vec{A}_\ind{\vec i_\ind{1},\bidot,\vec i_\ind{n}}$ $n$-D niza}
\glsentm{i}{ast}{\vec{A}_\ind{\vec i_1:\vec i_2}}
	{Podniz $\vec{A}_\ind{{\vec i_1}_\ind{1}:{\vec i_2}_\ind{1},\bidot,{\vec i_1}_\ind{n}:{\vec i_2}_\ind{n}}$ $n$-D niza}
\glsentm{i}{astp}{\vec{A}_\ind{\vec i_1:\vec i_2;:}}
	{Podniz $\vec{A}_\ind{{\vec i_1}_\ind{1}:{\vec i_2}_\ind{1},\bidot,{\vec i_1}_\ind{n-1}:{\vec i_2}_\ind{n-1},:}$ $n$-D niza}

% Operacije linearne algebre i operacije s nizovima
\glsgroup{l}{Operacije linearne algebre i operacije s nizovima} {} 
\glsentm{l}{scalprod}{\braket{\vec a}{\vec b},\,\vec{a}^\tp\vec{b}}
	{Skalarni produkt}
\glsentm{l}{outprod}{\vec{a}\vec{b}^\tp}
	{Vanjski produkt}
\glsentm{l}{hadprod}{\vec a \odot \vec b}
	{Umnožak po elementima; Hadamardov produkt}
\glsentm{l}{haddiv}{\vec a \oslash \vec b}
	{Dijeljenje po elementima}
\glsentm{l}{hadpow}{\vec a^{\odot b}}
	{Potenciranje po elementima}
\glsentm{l}{matmul}{\vec A \vec B}
	{Matrično množenje}
\glsentm{l}{matinv}{\vec A^{-1}}
	{Inverz matrice}
\glsentm{l}{transp}{\vec A^\tp}
	{Transponiranje}
\glsentm{l}{diag}{\diag\del{\vec{a}}}
	{Dijagonalna matrica kojoj dijagonalu čini vektor $\vec a$}
\glsentm{l}{det}{\det(\vec{A})}
	{Determinanta matrice $\vec A$}
\glsentm{l}{vecl2norm}{\enVert{\vec a}_2}
	{$\const L^2$-norma vektora $\vec a$}
\glsentm{l}{vecnorm}{\enVert{\vec a}_p}
	{$\const L^p$-norma vektora $\vec a$}
\glsentm{l}{matnorm}{\enVert{\vec A}_p}
	{Matrična $\const L^p$-norma matrice $\vec A$}
\glsentm{l}{frobnorm}{\enVert{\vec A}_\text{F}}
	{Frobeniusova norma matrice $\vec A$}
%\glsentm{l}{func}{f(\vec a)}
%	{Ako $f$ nije drugačije definirana i inače označava funkciju $\R\to\R$, onda se primjenjuje po elementima}
\glsentm{l}{conc}{\vec a\concat\vec b}
	{Konkatenacija vektora (stupaca) $\vec a\in\R^n$ i $\vec b\in\R^m$ u vektor iz $\R^{n+m}$}
\glsentm{l}{conc1}{\vec A\concat\vec B}
	{Konkatenacija nizova po prvoj dimenziji}
%\glsentm{l}{dconc}{\vec A\concat'\vec B}
%	{Konkatenacija nizova po zadnjoj dimenziji}
\glsentm{l}{vec}{\vecfunc(\vec A)}
	{Funkcija koja preslikava niz iz $\R^{d_1\times\dots\times d_n}$ u $\R^{d_1\dots d_n}$}
\glsentm{l}{vdim}{\dim(\vec a)}
	{Dimenzija vektora}
\glsentm{l}{dim}{\dim(\vec A)}
	{Vektor dimenzija niza; $\sbr{d_1,\bidot,d_n}$ za $\vec A\in\R^{d_1\times\dots\times d_n}$}

% Diferencijalni račun
\glsgroup{d}{Diferencijalni račun}{}
\glsentm{d}{od}{\tod{y}{x},\,\tod{}{x}f(x)}
	{Derivacija $y=f(x)$ po $x$}
\glsentm{d}{pd}{\tpd{y}{x},\,\tpd{}{x}f(x)}			
	{Parcijalna derivacija $y=f(x)$ po $x$}
\glsentm{d}{grad}{\nabla_{\vec x}{y},\,\nabla_{\vec x}{f(x)},\,\del{\tpd{y}{\vec x}}^\tp} 	
	{Gradijent $y=f(\vec x)$ po $\vec x$}
\glsentm{d}{gradmat}{\nabla_{\vec X}{y},\,\nabla_{\vec X}{f(x)}}	
	{Gradijent $y=f(\vec x)$ po $\vec X$}
\glsentm{d}{hess}{\tfrac{\partial^2y}{\partial\vec x\partial\vec x^\tp},\,\vec H_{f}(\vec x),\,\vec H}
	{Hessijan iz $\R^{n\times n}$ za $\funcdef{f}{\R^n}{\R}$ i $y=f(\vec x)$}
\glsentm{d}{jacobi}{\tpd{\vec y}{\vec x},\,\vec J_{f}(\vec x),\,\vec J}
	{Jakobijeva matrica iz $\R^{m\times n}$ za $\funcdef{f}{\R^n}{\R^m}$ i $\vec y=f(\vec x)$}
\glsentm{d}{int}{\int_{\set A}f(x)\dif x,\,\int_{x\in\set A}f(x)}
	{Određeni integral funkcije $f(x)$ po $x\in\set A$}
\glsentm{d}{int2}{\int f(x)\dif x,\,\int_x f(x)} 
	{Određeni integral funkcije $f(x)$ po $x\in\set A$, gdje je $\set A$ implicitan}

% Teorija vjerojatnosti
\glsgroup{tv}{Teorija vjerojatnosti}
{Svakoj slučajnoj varijabli $\rvar a$ jednoznačno je dodijeljena jedna razdioba $\p(\rvar a)$ (ili $\P(\rvar a)$) i funkcija gustoće vjerojatnosti (koja može biti poopćena funkcija) $p_{\rvar a}(a)=\p(\rvar a=a)$. $\P(A)$ označava vjerojatnost događaja $A$, a $P_{\rvar a}$ funkciju vjerojatnosti slučajne varijable $\rvar a$. Mogući su i kraći zapisi $\p(a)$ i $\P(a)$, gdje se po slovu koje označava vrijednost pretpostavlja slučajna varijabla označena istim slovom bez serifa. Mogu se koristiti i druge oznake za funkciju vjerojatnosti ili funkciju gustoće vjerojatnosti.}
%TODO move to new page if too high
\glsentm{tv}{rvarcond}{(\rvar a\mid\rvar b= b),\,(\rvar a\mid b)}{Uvjetna slučajna varijabla}
\glsentm{tv}{rvarjoint}{(\rvar a,\rvar b)}{Združena slučajna varijabla}
\glsentm{tv}{indep}{\rvar a\perp\rvar b}{\textit{Slučajne varijable $\rvar a$ i $\rvar b$ su nezavisne}}
\glsentm{tv}{dep}{\rvar a\not\perp\rvar b}{\textit{Slučajne varijable $\rvar a$ i $\rvar b$ su zavisne}}
\glsentm{tv}{condindep}{\rvar a\perp\rvar b\mid\rvar c}{\textit{Slučajne varijable $\rvar a$ i $\rvar b$ su uvjetno nezavisne uz poznat ishod slučajne varijable $\rvar c$}}
\glsentm{tv}{conddep}{\rvar a\not\perp\rvar b\mid\rvar c}{\textit{Slučajne varijable $\rvar a$ i $\rvar b$ su uvjetno zavisne uz poznat ishod slučajne varijable $\rvar c$}}
\glsentm{tv}{distr}{p,\,q}{Razdioba ili funkcija gustoće vjerojatnosti}
\glsentm{tv}{event}{\set A}{Događaj}
\glsentm{tv}{eventpred}{\cbr{R(\rvar a)}} {Događaj definiran predikatorm slučajne varijable $\rvar a$}
\glsentm{tv}{prob}{\P(\cbr{R(\rvar a)}),\,\P(R(\rvar a))} {Vjerojatnost događaja $\cbr{R(\rvar a)}$}
\glsentm{tv}{distrrvar}{\P(\rvar a),\,\p(\rvar a),\,\mathcal{D}} {Razdioba slučajne varijable $\rvar a$; $\P$ ako je $\rvar a$ diskretna slučajna varijabla, a $\p$ ako nije ili ako se ne zna}
\glsentm{tv}{probel}{\P(\rvar a= a),\,P_{\rvar a}(a),\,\P(a)} {Vjerojatnost događaja $\cbr{\rvar a=a}$}
\glsentm{tv}{probdens}{\p(\rvar a= a),\,p_{\rvar a}(a),\,\p(a)} {Gustoća vjerojatnosti događaja $\cbr{\rvar a=a}$}
\glsentm{tv}{pdfcond}{p_{\rvar a\mid b}(a),\,\p(a\mid b)} {Gustoća vjerojatnosti događaja $\cbr{\rvar a=a\mid\rvar b=b}$}
\glsentm{tv}{pdfjoint}{p_{\rvar a,\rvar b}(a,b),\,\p(a,b)} {Gustoća vjerojatnosti događaja $\cbr{\rvar a=a,\rvar b=b}$}
\glsentm{tv}{hasdistrib}{\rvar a \sim q,\, \p(\rvar a)=q} {\textit{Slučajna varijabla $\rvar a$ ima razdiobu $q$}}
\glsentm{tv}{hasdistribset}{\rvar a \sim \set A} 	{\textit{Slučajna varijabla $\rvar a$ ima takvu razdiobu da svi elementi (multi)skupa $\set A$ imaju vjerojatnost proporcionalnu višestrukosti ($\frac{1}{\envert{\set A}}$ za običan skup)}}
\glsentm{tv}{fromdistrib}{a\sim q} {\textit{$a$ se izvlači iz razidiobe $q$}}
\glsentm{tv}{fromrvar}{a\sim \rvar a,\,a\sim \p(\rvar a)} {\textit{$a$ se izvlači iz razidobe $\p(\rvar a)$}}
\glsentm{tv}{E}{\E_{a\sim\rvar a} f(a),\,\E_{\rvar a} f(a)} {Očekivanje funkcije slučajne varijable $\rvar a$}
\glsentm{tv}{D}{\D_{a\sim\rvar a} f(a),\,\D_{\rvar a} f(a)} {Disperzija (varijanca) funkcije slučajne varijable $\rvar a$}
\glsentm{tv}{Cov}{\Cov(\rvar a,\rvar b)}		{Kovarijanca}
\glsentm{tv}{Gauss}{\mathcal{N}(\mu, \sigma^2)} {Normalna razdioba s učekivanjem $\mu$ i varijancom $\sigma^2$}
\glsentm{tv}{unif}{\mathcal{U}(\set A)}
	{Uniformna razdioba nad skupom $\set A$}

% Information theory
\glsgroup{ti}{Information theory}{}
\glsentm{ti}{I}{\I(\set A)}
	{Information content of event $\set A$}
\glsentm{ti}{entropy}{\H(\rvar a)}
	{Entropy}
\glsentm{ti}{diffent}{\h(\rvar a)}
	{Differnetial entropy}
\glsentm{ti}{mutinf}{\I(\rvar a;\rvar b)}
	{Mutual information}
\glsentm{ti}{condent}{\H(\rvar a\mid\rvar b)}
	{Conditional entropy}
\glsentm{ti}{crossent}{\H_{\rvar b}(\rvar a)}
	{Cross entropy}
\glsentm{ti}{relent}{\Dist_{\rvar b}(\rvar a)}
	{Relative entropy (Kullback-Leibler divergence)}

% Grafovi
\glsgroup{g}{Grafovi}{}
\glsentm{g}{pa}{\pa_G(a)}
	{Skup čvorova roditelja čvora $a$ u grafu $G$}
\glsentm{g}{ch}{\ch_G(a)}
	{Skup čvorova djece čvora $a$ u grafu $G$}
\glsentm{g}{pred}{\pred_G(a)}
	{Skup čvorova prethodnika čvora $a$ u grafu $G$}
\glsentm{g}{succ}{\succ_G(a)}
	{Skup čvorova nasljednika čvora $a$ u grafu $G$}

% Ostalo
\glsgroup{f}{Other}{}
\glsentm{f}{funcset}{\set A\to\set B}
	{A set of functions with domain $\set A$ and codomain $\set B$}
%\glsentm{f}{func}{\funcdef{f}{\set A}{\set B}}
%	{Funkcija s domenom $\set A$ i kodomenom $\set B$}
%\glsentm{f}{fusnc}{f\in(\set A\to\set B)}
%	{Funkcija s domenom $\set A$ i kodomenom $\set B$}
\glsentm{f}{funcdefs}{f\in(\set A\to\set B)}
	{Function definition; a function mapping from its domain $\set A$ to its codomain $\set B$}
\glsentm{f}{funcdef}{x\mapsto f(x)}
	{Function definition; a function mapping $x$ from its domain to $f(x)$ in its codomain}
\glsentm{f}{fadd}{f+g}
	{Sum of functions}
\glsentm{f}{fmul}{fg}
	{Product of functions}
\glsentm{f}{conv}{f*g}
	{Convolution of functions}
\glsentm{f}{fscalprod}{\braket{f}{g}}
	{Scalar product of functions}
\glsentm{f}{card}{\envert{\set A}}
	{Set cardinality}
\glsentm{f}{dirac}{\dirac\del{\cdot}}
	{Dirac delta}
\glsentm{f}{doublebracket}{\enbbracket{\cdot}}
	{Iverson bracket; $\enbbracket{P}=\begin{cases} 1, & P \equiv \top \\ 0, & P \equiv \bot \end{cases}$}
\glsentm{f}{image}{f\sbr{\set A}}
	{Image of $\set A$ (subset of the domain) under $f$; $f\sbr{\set A} \coloneqq \cbr{f(a)\mid a\in\set A}$}
\glsentm{f}{invimage}{f^{-1}\sbr{\set B}}
	{Inverse image (preimage) of $\set B$ (subset of the codomain) under $f$; $f^{-1}\sbr{\set B} \coloneqq \cbr{a\mid f(a)\in\set B}$}